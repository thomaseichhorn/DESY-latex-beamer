% Thomas Eichhorn 2017
% DESY-Style latex beamer

% Packages in Ubuntu 16.04:
% preview-latex-style texlive-latex-extra texlive-latex-extra-doc texlive-science texlive-science-doc texlive-bibtex-extra texlive-fonts-extra texlive-fonts-extra-doc texlive-fonts-recommended texlive-fonts-recommended-doc

\documentclass{beamer}

% Some packages...
\usepackage{graphicx}
\usepackage[english]{babel}
\usepackage[latin1]{inputenc}
\usepackage{times}
\usepackage[T1]{fontenc}
\usepackage{lipsum}
\usepackage{hyperref}
\usepackage{comment}
\usepackage[amssymb]{SIunits}
\usepackage{amssymb,amsfonts,amsmath}
\usepackage[default]{gillius}
\usepackage{array}
\usepackage{multirow}
\usepackage{marvosym}
\usepackage{tikz}
\usetikzlibrary{plotmarks}
\usepackage{caption}
\usepackage{bookmark}
\usepackage{pgfplots}
\usepackage{sidecap}
\usepackage{cancel}
%\usepackage{subfigure}
\usepackage{subcaption}
\usepackage{hepnames}
\usepackage{picture}

% Allow tiny captions
\DeclareCaptionFont{tiny}{\tiny}

% A gaussian to be used in a tikz pic
\pgfmathdeclarefunction{gauss}{3}{\pgfmathparse{#3/(#2*sqrt(2*pi))*exp(-((x-#1)^2)/(2*#2^2))}}

% A backup for all those questions...
\newcommand{\backupbegin}{\newcounter{finalframe}\setcounter{finalframe}{\value{framenumber}}}
\newcommand{\backupend}{\setcounter{framenumber}{\value{finalframe}}}

% The style definitions
\mode<presentation>
{
    \usetheme{CambridgeUS}
    \setbeamercovered{transparent}
    \definecolor{desyblue}{RGB}{0,166,235}
    \definecolor{desyorange}{RGB}{242,142,0}
    \definecolor{desygrey}{RGB}{119,119,119}
    \definecolor{desylight}{RGB}{233,233,233}
    \definecolor{darkgreen}{RGB}{0,128,0}
    \definecolor{darkblue}{RGB}{0,0,128}
    \setbeamercolor{normal text}{fg=black,bg=white}
    \setbeamercolor{alerted text}{fg=desyorange}
    \setbeamercolor{example text}{fg=desyblue}
    \setbeamercolor{structure}{fg=desyblue}
    \setbeamercolor{local structure}{fg=desyorange}
    \setbeamercolor{title}{fg=white,bg=desyblue}
    \setbeamercolor{subtitle}{fg=desyorange,bg=white}
    \setbeamercolor{author}{fg=desyblue}
    \setbeamercolor{date}{fg=black}
    \setbeamercolor{institute}{fg=black}
    \setbeamercolor{title in head/foot}{fg=desygrey,bg=white}
    \setbeamercolor{author in head/foot}{fg=desygrey,bg=white}
    \setbeamercolor{institute in head/foot}{fg=desygrey,bg=white}
    \setbeamercolor{subtitle in head/foot}{fg=desygrey,bg=white}
    \setbeamercolor{pagenumber in head/foot}{fg=desygrey,bg=white}
    \setbeamercolor{date in head/foot}{fg=desygrey,bg=white}
    \setbeamercolor{frametitle}{fg=white,bg=desyblue}
    \setbeamercolor{section in head/foot}{fg=white,bg=desyblue}
    \setbeamercolor{subsection in head/foot}{fg=white,bg=desyblue}
    \setbeamercolor{subsubsection in head/foot}{fg=white,bg=desyblue}
    \setbeamercolor{subsection number projected}{fg=black,bg=desygrey}
    \setbeamercolor{the unknown}{parent=local structure}
    \setbeamercolor{palette primary}{use=structure,fg=structure.fg}
    \setbeamercolor{palette secondary}{use=structure,fg=structure.fg!75!black}
    \setbeamercolor{palette tertiary}{use=local structure,fg=local structure.fg}
    \setbeamercolor{palette quaternary}{fg=black}
    \setbeamercolor{block body}{fg=black,bg=desyblue!10!white}
    \setbeamercolor{block body alerted}{}
    \setbeamercolor{block body example}{}
    \setbeamercolor{block title}{fg=white,bg=desyblue}
    \setbeamercolor{block title alerted}{parent=alerted text}
    \setbeamercolor{block title example}{parent=example text}
    \setbeamercolor{bibliography entry author}{fg=black}
    \setbeamercolor{bibliography entry title}{fg=structure}
    \setbeamercolor{bibliography entry location}{fg=black}
    \setbeamercolor{bibliography entry note}{fg=black}
    \setbeamercolor{footnote}{fg=desygrey}
    \setbeamercolor{footnote mark}{fg=desygrey}
    \setbeamercolor{subsection in toc shaded}{fg=desygrey}
    \setbeamercolor{frame}{fg=black}
    \setbeamerfont{structure}{series=\bfseries}
    \setbeamerfont{local structure}{series=\bfseries}
    \setbeamerfont{block title}{size={}}
    \setbeamerfont{tiny structure}{series=\bfseries}
}

% Navigation symbols, we don't need all of them
\setbeamertemplate{navigation symbols}{\insertslidenavigationsymbol \insertframenavigationsymbol }

% Header, should be big enough
\setbeamertemplate{headline}
{
    \leavevmode%
    \hbox{%
	\begin{beamercolorbox}[wd=.5\paperwidth,ht=2.1ex,dp=1ex,center]{section in head/foot}%
	\usebeamerfont{section in head/foot}\insertsectionhead\hspace*{1ex}
	\end{beamercolorbox}%
	\begin{beamercolorbox}[wd=.5\paperwidth,ht=2.1ex,dp=1ex,center]{subsection in head/foot}%
	    \usebeamerfont{subsection in head/foot}\hspace*{1ex}\insertsubsectionhead
	\end{beamercolorbox}}%
    \vskip0pt%
}

% Frametitle
\setbeamertemplate{frametitle}
{
    \vspace{-0.4ex}
    \begin{beamercolorbox}[sep=0.1ex, wd=\paperwidth]{frametitle}
	\usebeamercolor[bg]{frametitle}\usebeamercolor[fg]{frametitle}\usebeamerfont{frametitle}
	\strut\insertframetitle\strut
	\vspace{-0.1ex}
    \end{beamercolorbox}
    \vspace{-1.5ex}
}

% Toc style with shading to indicate position
\setbeamertemplate{section in toc shaded}
{
    \begin{colormixin}{46.6667!parent.bg}
    \usebeamertemplate{section in toc}
    \end{colormixin}\unskip
}
\setbeamertemplate{subsection in toc shaded}
{
    \begin{colormixin}{46.6667!parent.bg}
    \usebeamertemplate{subsection in toc}
    \end{colormixin}\unskip
}
\setbeamertemplate{subsubsection in toc shaded}
{
    \begin{colormixin}{46.6667!parent.bg}
    \usebeamertemplate{subsubsection in toc}
    \end{colormixin}\unskip
}
\setbeamertemplate{section in toc}
{
    \usebeamerfont{section in toc}
    \usebeamercolor[fg]{local structure}\textgreater\
    \usebeamercolor[fg]{structure}\inserttocsection\par
}
\setbeamertemplate{subsection in toc}[square]
\setbeamertemplate{subsubsection in toc}{}

% Items
\setbeamertemplate{items}{\textgreater}
\setbeamertemplate{itemize subitem,itemize subsubitem}[square]
\setbeamertemplate{bibliography item}{\textgreater}

% Footline
\setbeamertemplate{sidebar right}{}
\setbeamertemplate{footline}{
    \vspace{-3.5ex}
    \hfill
    \begin{tabular}{m{.85\textwidth} m{.075\textwidth}}
	% other logos (e.g. your uni) go here...
	%\includegraphics[width=.045\textwidth]{logos/uhh.pdf}
	\vspace{4ex}
	\hfill
	\usebeamertemplate***{navigation symbols}
	\hspace{2ex}
	\usebeamercolor[fg]{author in head/foot}\usebeamerfont{author in head/foot}\insertshortauthor\ $|$
	\usebeamercolor[fg]{title in head/foot}\usebeamerfont{title in head/foot}\insertshorttitle\ $|$
	\insertshortdate\ $|$
	\usebeamercolor[fg]{pagenumber in head/foot} \usebeamerfont{pagenumber in head/foot} Page \insertframenumber{} of \inserttotalframenumber
	& \includegraphics[width=.054\textwidth]{logos/desy.pdf}
    \end{tabular}
}

% Boxes with shadows around them
\makeatletter
\newcommand\beamerboxesframed[2][]{%
    \global\let\beamer@firstlineitemizeunskip=\relax%
    \vbox\bgroup%
    \setkeys{beamerboxes}{upper=block title,lower=block body,width=\textwidth}%
    \setkeys{beamerboxes}{#1}%
    {%
	\usebeamercolor{\bmb@lower}%
	\globalcolorstrue%
	\colorlet{lower.bg}{bg}%
    }%
    {%
	\usebeamercolor{\bmb@upper}%
	\globalcolorstrue%
	\colorlet{upper.bg}{bg}%
    }%
    % Typeset head
    \vskip4bp
    \setbox\bmb@box=\hbox{%
    \begin{minipage}[b]{\bmb@width}%
	\usebeamercolor[fg]{\bmb@upper}%
	#2%
    \end{minipage}}%
    \ifdim\wd\bmb@box=0pt%
	\setbox\bmb@box=\hbox{}%
	\ht\bmb@box=0pt%
	\bmb@prevheight=-4.5pt%
    \else%
	\wd\bmb@box=\bmb@width%
	\bmb@temp=\dp\bmb@box%
	\ifdim\bmb@temp<1.5pt%
	    \bmb@temp=1.5pt%
	\fi%
	\setbox\bmb@box=\hbox{\raise\bmb@temp\hbox{\box\bmb@box}}%
	\dp\bmb@box=0pt%
	\bmb@prevheight=\ht\bmb@box%
    \fi%
    \bmb@temp=\bmb@width%
    \bmb@dima=\bmb@temp\advance\bmb@dima by2.2bp%
    \bmb@dimb=\bmb@temp\advance\bmb@dimb by4bp%
    \hbox{%
	\begin{pgfpicture}{0bp}{+-\ht\bmb@box}{0bp}{+-\ht\bmb@box}
	    \ifdim\wd\bmb@box=0pt%
		\color{lower.bg}%
	    \else%
		\color{upper.bg}%
	    \fi%
	    \pgfpathqmoveto{-4bp}{-1bp}
	    \pgfpathqcurveto{-4bp}{1.2bp}{-2.2bp}{3bp}{0bp}{3bp}
	    \pgfpathlineto{\pgfpoint{\bmb@temp}{3bp}}
	    \pgfpathcurveto%
	    {\pgfpoint{\bmb@dima}{3bp}}%
	    {\pgfpoint{\bmb@dimb}{1.2bp}}%
	    {\pgfpoint{\bmb@dimb}{-1bp}}%
	    \bmb@dima=-\ht\bmb@box%
	    \advance\bmb@dima by-2pt%
	    \pgfpathlineto{\pgfpoint{\bmb@dimb}{\bmb@dima}}
	    \pgfpathlineto{\pgfpoint{-4bp}{\bmb@dima}}
	    \pgfpathclose
	    \pgfsetstrokecolor{black}\pgfusepath{stroke, fill}
	\end{pgfpicture}%
	\copy\bmb@box%
    }%
    \nointerlineskip%
    \ifdim\wd\bmb@box=0pt
    \else
	\vskip2.4pt%
    \fi%
    \nointerlineskip%
    \setbox\bmb@colorbox=\hbox{{\pgfpicturetrue\pgfsetcolor{lower.bg}}}%
    \setbox\bmb@box=\hbox\bgroup\begin{minipage}[b]{\bmb@width}%
    \vskip2pt%
    \usebeamercolor[fg]{\bmb@lower}%
    \colorlet{beamerstructure}{upper.bg}%
    \colorlet{structure}{upper.bg}%
    %\color{.}%
}

\def\endbeamerboxesframed{%
    \end{minipage}\egroup%
    \wd\bmb@box=\bmb@width%
    \bmb@temp=\dp\bmb@box%
    \advance\bmb@temp by.5pt%
    \setbox\bmb@box=\hbox{\raise\bmb@temp\hbox{\box\bmb@box}}%
    \dp\bmb@box=0pt%
    \bmb@temp=\wd\bmb@box%
    \bmb@dima=\bmb@temp\advance\bmb@dima by2.2bp%
    \bmb@dimb=\bmb@temp\advance\bmb@dimb by4bp%
    \hbox{%
	\begin{pgfpicture}{0bp}{0bp}{0bp}{0bp}
	    \unhbox\bmb@colorbox%
	    \pgfpathmoveto{\pgfpoint{-4bp}{\ht\bmb@box}}
	    \pgfpathlineto{\pgfpoint{-4bp}{1bp}}
	    \pgfpathqcurveto{-4bp}{-1.2bp}{-2.2bp}{-3bp}{0bp}{-3bp}
	    \pgfpathlineto{\pgfpoint{\the\bmb@temp}{-3bp}}
	    \pgfpathcurveto%
	    {\pgfpoint{\the\bmb@dima}{-3bp}}%
	    {\pgfpoint{\the\bmb@dimb}{-1.2bp}}%
	    {\pgfpoint{\the\bmb@dimb}{1bp}}%
	    {
		\bmb@dima=\ht\bmb@box%
		\pgfpathlineto{\pgfpoint{\bmb@dimb}{\bmb@dima}}
		\pgfsetstrokecolor{black}\pgfusepath{stroke, fill}
	    }
	\end{pgfpicture}%
	\box\bmb@box%
    }%
    \vskip2bp%
    \egroup% of \vbox\bgroup
}
\makeatother

% The blocks
\defbeamertemplateparent{blocks}[framed]{block begin,block end,%
block alerted begin,block alerted end,%
 block example begin,block example end}[1][]
{[#1]}

\defbeamertemplate{block begin}{framed}[1][]
{
    \par\vskip\medskipamount%
    \begin{beamerboxesframed}[upper=block title,lower=block body,#1]%
    {\raggedright\usebeamerfont*{block title}\insertblocktitle}%
    \raggedright%
    \usebeamerfont{block body}%
}
\defbeamertemplate{block end}{framed}[1][]
{\end{beamerboxesframed}\vskip\smallskipamount}

\defbeamertemplate{block alerted begin}{framed}[1][]
{
    \par\vskip\medskipamount%
    \begin{beamerboxesframed}[upper=block title alerted,lower=block body alerted,#1]%
    {\raggedright\usebeamerfont*{block title alerted}\insertblocktitle}%
    \raggedright%
    \usebeamerfont{block body alerted}%
}%
\defbeamertemplate{block alerted end}{framed}[1][]
{\end{beamerboxesframed}\vskip\smallskipamount}

\defbeamertemplate{block example begin}{framed}[1][]
{
    \par\vskip\medskipamount%
    \begin{beamerboxesframed}[upper=block title example,lower=block body example,#1]
    {\raggedright\usebeamerfont*{block title example}\insertblocktitle}%
    \raggedright%
    \usebeamerfont{block body alerted}%
}%
\defbeamertemplate{block example end}{framed}[1][]
{\end{beamerboxesframed}\vskip\smallskipamount}

\setbeamercolor{footnote}{fg=black}
\setbeamercolor{footnote mark}{fg=black}

% Alternative:
%\setbeamertemplate{blocks}[framed]

% original:
\begin{comment}
% the title page
\defbeamertemplate*{title page}{logos/desy.pdf}
{
\vspace{-12.0ex}
\begin{beamercolorbox}[sep=.5ex,wd=\paperwidth,leftskip=.5\paperwidth]{title}
 \usebeamercolor[fg]{title}\centering\usebeamerfont{title}\inserttitle\hspace{1mm}\usebeamercolor[fg]{the unknown}\huge\textrm{.}\par
\end{beamercolorbox}
%\begin{beamercolorbox}[leftskip=.6em, sep=.5ex, wd=\paperwidth]{subtitle}
%  \usebeamerfont{subtitle}\usebeamercolor[fg]{subtitle}\insertsubtitle\par
%\end{beamercolorbox}
%\vfill
\hfill\begin{beamercolorbox}[sep=.5ex, center, wd=\paperwidth]{titlegraphic}
 \inserttitlegraphic
\end{beamercolorbox}
\vfill
%\vskip8.5ex
%\vskip.5ex
\begin{beamercolorbox}[sep=.5ex,wd=\paperwidth,leftskip=.5\paperwidth]{author}
%\begin{beamercolorbox}[]{author}
 \usebeamercolor[fg]{author}\centering\usebeamerfont{author}\insertauthor\par
 \vspace{1mm}
 \usebeamercolor[fg]{institute}\usebeamerfont{institute} PhD Defense \par
 \vspace{1mm}
 \usebeamercolor[fg]{date}\usebeamerfont{date}\insertdate\par
\end{beamercolorbox}

}
\end{comment}

% Remove space around the blocks
\addtobeamertemplate{block begin}{\vspace*{-3pt}}{}
\addtobeamertemplate{block end}{}{\vspace*{-3pt}}

% Captions without figure X
\setbeamerfont{caption}{size=\footnotesize}
\captionsetup[figure]{labelformat=empty}
\setlength\abovecaptionskip{-1pt}

% Let's go!

\title[A Short Title]{A Long Title}

% Optional
\subtitle{A subtitle}

% You can have multiple people here
\author{My Name}

\date{January 1st, 2017}

% Delete this, if you do not want the table of contents to pop up at
% the beginning of each subsection:
\AtBeginSection[]
{
  \begin{frame}<beamer>{Outline}
    \tableofcontents[currentsection]
  \end{frame}
}

% If you wish to uncover everything in a step-wise fashion, uncomment the following command: 
%\beamerdefaultoverlayspecification{<+->}

\newenvironment<>{varblock}[2][\textwidth]{
    \begin{center}
      \begin{minipage}{#1}
        \setlength{\textwidth}{#1}
          \begin{actionenv}#3
            \def\insertblocktitle{#2}
            \par
            \usebeamertemplate{block begin}}
  {\par
      \usebeamertemplate{block end}
    \end{actionenv}
  \end{minipage}
\end{center}}


\begin{document}

% no head and footline on titlepage
{
\setbeamertemplate{headline}{}
{
\setbeamertemplate{footline}{}
\begin{frame}
  \titlepage
\end{frame}

}
}

% You can redefine the author variable, so only your name is on the foot of each slide
%\author{My Name}

% Count one back, so the titlepage doesn't count
\addtocounter{framenumber}{-1}

% PDF bookmarks
\bookmark[page=1,level=1]{Preface}

% A slide with the table of contents
% You might wish to add the option [pausesections]
\begin{frame}{Outline}
  \tableofcontents
\end{frame}


\section{Introduction}

\subsection{Some subtitle}

\begin{frame}
    \frametitle{A title of this frame}

    % columns, use [T] to align the blocks at the top
    \begin{columns}[T]

	% .46 gives a nice space so the page is not too wide
	\column{.46\textwidth}

	\begin{block}{A block title}
	    \begin{itemize}
		\item A list item
		\item A \alert{highlighted} list item
	    \end{itemize}
	\end{block}

	\column{.46\textwidth}

	\begin{block}{Another block title}
	    \begin{itemize}
		\item[\MVRightarrow{}] An item with an arrow
	    \end{itemize}
	\end{block}

	\vspace{25pt}

	\begin{block}{A spaced block}
	    \begin{itemize}
		\item An item
		\item More items
		\item ...
	    \end{itemize}
	\end{block}

    \end{columns}

\end{frame}

\subsection{Another subsection}

\begin{frame}
    \frametitle{An empty slide}

\end{frame}

\section{Another section}

\subsection{More subsections}

\begin{frame}
    \frametitle{A title}

    \begin{columns}

	\column{.46\textwidth}

	\begin{block}{asdf}
	    \begin{itemize}
		\item a
		\item s
		\item d
		\item f
	    \end{itemize}
	\end{block}

	\vspace{2ex}

	\begin{block}{Block with a table}
	    \centering
	    \vspace{1pt}
	    \begin{table}[h]\small
		\begin{tabular}{c|c}
		    a & b \\
		    \hline
		    $1$ & $2$ \\
		    $3$ & $4$ \\
		    \alert{$5$} & \alert{$6$}
		\end{tabular}
	    \end{table}
	    \vspace{-3ex}
	\end{block}

	\column{.46\textwidth}

        \begin{block}{Block}
	    \begin{itemize}
		\item a
	    \end{itemize}
	\end{block}

    \end{columns}

    \vspace{1ex}

    \begin{block}{A block outside of the columns}
	\begin{itemize}
	    \item a
	    \item b
	\end{itemize}
    \end{block}

\end{frame}

\begin{frame}
    \frametitle{A frame with a tikz picture}

    \begin{columns}[T]

	\column{.46\textwidth}

	\begin{block}{Block}
	    \begin{itemize}
		\item a
		\item b
	    \end{itemize}
	\end{block}

	\column{.46\textwidth}

	\begin{figure}
	    \centering
	    \begin{tikzpicture}[scale=0.3]
		\begin{axis}[every axis plot post/.append style={
		    mark=none,domain=-1:3,samples=50,smooth}, % All plots: from -2:2, 50 samples, smooth, no marks
		    axis x line=bottom, % no box around the plot, only x and y axis
		    axis y line*=left, % the * suppresses the arrow tips
		    enlargelimits=upper, % extend the axes a bit to the right and top
		    yticklabels={,,},
		    xticklabels={,,},
		    x tick label style={major tick length=0pt},
		    y tick label style={major tick length=0pt}]
		    \addplot[color=red,fill=red, fill opacity = 0.2] {gauss(0,0.3,1)};
		    \addplot[color=blue,fill=blue, fill opacity = 0.2] {gauss(1,0.5,1.5)};
		\end{axis}
		\node[->,red](title) at (1.5,-1,0){\footnotesize \rotatebox{0}{$\textrm{Noise}$}};
		\node[->,blue](title) at (7,2,0){\footnotesize \rotatebox{0}{$\textrm{Signal}$}};
		\node[->,blue](title) at (5.5,5.5,0){\footnotesize \rotatebox{0}{$\Phi$}};
		\draw[->,blue,thick] (5,5.5,0) -- (3,5.5,0);
		\node[->,black](title) at (7.5,-0.5,0){\tiny \rotatebox{0}{$\textrm{Q}$}};
	    \end{tikzpicture}
	\end{figure}

    \end{columns}

\end{frame}

\begin{frame}
    \frametitle{A frame with uncovering}

    \begin{columns}

	\column{.46\textwidth}

	\begin{block}{First block, stays}<1->
	    \begin{itemize}
		\item a
		\item b
		\item c
	    \end{itemize}
	\end{block}

	\vspace{15pt}

	\begin{block}{Second block, hidden again}<2-2>
	    \begin{itemize}
		\item a
		\item b
	    \end{itemize}
	\end{block}

	\column{.46\textwidth}

	\begin{block}{Third block}<3->
	    \begin{itemize}
		\item a
		\item b
		\item c
	    \end{itemize}
	\end{block}

    \end{columns}

\end{frame}

\begin{frame}
    \frametitle{A frame with an overlayarea}

    \begin{columns}[T]

	\column{.46\textwidth}

	\begin{block}{Block one}<1>
	    \begin{itemize}
		\item a
	    \end{itemize}
	\end{block}

	\begin{block}{Block 2}<2->
	    \begin{itemize}
		\item b
	    \end{itemize}
	\end{block}

	\column{.46\textwidth}

	\begin{overlayarea}{\textwidth}{.75\textheight}

	\only<1>{
	    \begin{figure}[hbtp]
		\centering
		\captionsetup{font=tiny}
		\caption{A caption}
		\vspace{5pt}
		\includegraphics[trim=0 0 0 0,clip,width=0.6\textwidth]{logos/desy.pdf}
	    \end{figure}

	    \begin{block}{}
		\begin{itemize}
		    \item A title-less block describing the picture
		\end{itemize}
	    \end{block}
	}

	\only<2>{
	    \begin{figure}[hbtp]
		\centering
		\captionsetup{font=tiny}
		\caption{Another caption}
		\vspace{5pt}
		\includegraphics[trim=0 0 0 0,clip,width=0.6\textwidth]{logos/desy.pdf}
	    \end{figure}

	    \begin{block}{}
		\begin{itemize}
		    \item a
		\end{itemize}
	    \end{block}
	}

	\end{overlayarea}

    \end{columns}

\end{frame}

\appendix
\backupbegin
\section*{Backup Slides}

\begin{frame}
  \frametitle{Backup Slides}

\end{frame}

\begin{frame}
    \frametitle{A title of this frame}

    % columns, use [T] to align the blocks at the top
    \begin{columns}[T]

	% .46 gives a nice space so the page is not too wide
	\column{.46\textwidth}

	\begin{block}{A block title}
	    \begin{itemize}
		\item A list item
		\item A \alert{highlighted} list item
	    \end{itemize}
	\end{block}

	\column{.46\textwidth}

	\begin{block}{Another block title}
	    \begin{itemize}
		\item[\MVRightarrow{}] An item with an arrow
	    \end{itemize}
	\end{block}

	\vspace{25pt}

	\begin{block}{A spaced block}
	    \begin{itemize}
		\item An item
		\item More items
		\item ...
	    \end{itemize}
	\end{block}

    \end{columns}

\end{frame}

\backupend

\end{document}



